\documentclass[nohyper,nobib]{tufte-handout}

\usepackage[%
  backend      = biber,
  doi          = true,
  url          = true,
  giveninits   = true,
  hyperref     = true,
  maxbibnames  = 99,
  maxcitenames = 99,
  sortcites    = true,
  sorting      = none,
  style        = numeric,
]{biblatex}

\usepackage[english]{babel} % English language/hyphenation
\usepackage[T1]{fontenc} %
\usepackage[utf8]{inputenc} %
\usepackage{microtype} % Font protrusion and expansion

\usepackage{amsmath,amsfonts,amsthm,amssymb} % Math packages
\usepackage{physics} %
\usepackage{unicode-math} %

\usepackage{graphicx} % allow embedded images
  \setkeys{Gin}{width=\linewidth,totalheight=\textheight,keepaspectratio}
  \graphicspath{{graphics/}} % set of paths to search for images

\usepackage{booktabs} % book-quality tables
\usepackage{multicol} % multiple column layout facilities
\usepackage{fancyvrb} % extended verbatim environments
\fvset{fontsize=\normalsize}% default font size for fancy-verbatim environments
\usepackage{fancyhdr} % Custom headers and footers
\pagestyle{fancyplain} % Makes all pages in the document conform to the custom headers and footers
\fancyhead{} % No page header - if you want one, create it in the same way as the footers below
\fancyfoot[L]{} % Empty left footer
\fancyfoot[C]{} % Empty center footer
\fancyfoot[R]{\thepage} % Page numbering for right footer
\renewcommand{\headrulewidth}{0pt} % Remove header underlines
\renewcommand{\footrulewidth}{0pt} % Remove footer underlines
\setlength{\headheight}{13.6pt} % Customize the height of the header
\allowdisplaybreaks

\usepackage{float}
\usepackage{caption}
\usepackage{verbatim}
\usepackage{sectsty} % Allows customizing section commands
\allsectionsfont{\normalfont \bfseries} % Make all sections centered, the default font and small caps
\usepackage{enumerate}

\usepackage{color} % colors
\usepackage[dvipsnames]{xcolor}
\usepackage[colorlinks=true, linkcolor=WildStrawberry, urlcolor=WildStrawberry, citecolor=Emerald, linktocpage=true, breaklinks=true]{hyperref}

\usepackage{varioref}
\usepackage{xspace}

%%
% Figures
\newcommand{\labelFigure}[1]{%
  % #1 = label
  \label{fig:#1}%
}
\newcommand{\refFigure}[1]{%
  % #1 = reference
  \figurename~\vref{fig:#1}\xspace%
}

\newcommand{\refFigureOnly}[1]{%
  % #1 = reference
  \figurename~\ref{fig:#1}\xspace%
}

\newcommand{\refSubfigureOnly}[1]{%
  % #1 = reference
  \subcaptionref{fig:#1}\xspace%
}

%%
% Tables
\newcommand{\labelTable}[1]{%
  % #1 = label
  \label{tab:#1}%
}
\newcommand{\refTable}[1]{%
  % #1 = reference
  \tablename~\vref{tab:#1}\xspace%
}
\newcommand{\refTableOnly}[1]{%
  % #1 = reference
  \tablename~\ref{tab:#1}\xspace%
}
\newcommand{\refSubtableOnly}[1]{%
  % #1 = reference
  \subcaptionref{tab:#1}\xspace%
}

%%
% Sections
\newcommand{\labelSection}[1]{%
  % #1 = label
  \label{sec:#1}%
}
\newcommand{\refSection}[1]{%
  % #1 = reference
  §~\vref{sec:#1}\xspace%
}
\newcommand{\refSectionOnly}[1]{% Doesn't include the page referencing
  % #1 = reference
  §~\ref{sec:#1}\xspace%
}


%--------Theorem Environments--------
%theoremstyle{plain} --- default
\newtheorem{thm}{Theorem}
\newtheorem{cor}[thm]{Corollary}
\newtheorem{prop}[thm]{Proposition}
\newtheorem{facts}[thm]{Facts}
\newtheorem{fact}[thm]{Fact}
\newtheorem{clm}[thm]{Claim}
\newtheorem{lem}[thm]{Lemma}
\newtheorem{conj}[thm]{Conjecture}
\newtheorem{quest}[thm]{Question}

\theoremstyle{definition}
\newtheorem{defn}[thm]{Definition}
\newtheorem{defns}[thm]{Definitions}
\newtheorem{con}[thm]{Construction}
\newtheorem{exmp}[thm]{Example}
\newtheorem{exmps}[thm]{Examples}
\newtheorem{notn}[thm]{Notation}
\newtheorem{notns}[thm]{Notations}
\newtheorem{addm}[thm]{Addendum}
\newtheorem{exer}[thm]{Exercise}

\theoremstyle{remark}
\newtheorem{rem}[thm]{Remark}
\newtheorem{rems}[thm]{Remarks}
\newtheorem{warn}[thm]{Warning}
\newtheorem{sch}[thm]{Scholium}

\defaultfontfeatures{Scale=MatchLowercase, Ligatures=TeX}
\setmainfont{EB Garamond}
\setsansfont{URW Classico}
\setmathfont{Garamond-Math}
\setmonofont{Inconsolata}

\geometry{
	left=13mm, % left margin
	textwidth=130mm, % main text block
	marginparsep=8mm, % gutter between main text block and margin notes
	marginparwidth=55mm % width of margin notes
}

\fontsize{20}{10}\selectfont

%----------------------------------------------------------------------------------------
%	TITLE SECTION
%----------------------------------------------------------------------------------------

\title{Notes on Charles Pinter's `A Book Of Abstract Algebra'} % The assignment title
\author{Unathi Skosana}
\date{\vspace{-5pt}\normalsize \today} % Today's date or a custom date

\begin{document}
\justifying 
\maketitle

\begin{abstract}
\noindent
Personal notes taken while studying Charles Pinter's `A Book Of Abstract Algebra'
\end{abstract}

\tableofcontents

\section{Operations}

\begin{quest}
	What is an \textbf{operation} on a set A?
\end{quest}

\begin{defns}[Informal definition]
    An operation is any rule which assigns to each ordered pair of elements of $A$ a unique element in $A$.
\end{defns}

\begin{defns}[Formal definition]
	Let $A$ be any set:\newline An operation $*$ on $A$ is a rule which assigns to each ordered pairs $(a,b)$ of elements of $A$ exactly one $a * b$ in $A$, such that:
    \begin{itemize}
        \item $a * b$ is defined for \emph{every} ordered pair $(a,b)$ of elements of $A$. \footnote{In $\mathbb{R}$, division does not qualify as operation since it does not satisfy this condition. i.e. the ordered pair $(a, 0)$ has undefined quotient $a / 0$.}
        \item $a * b$ must be \emph{uniquely} defined. \footnote{If $\diamond$ is defined on $(a, b)$ to be the number whose square is $ab$. In $\mathbb{R}$, $\diamond$ does not qualify as an operation since $2 \diamond 2$  could be either $2$, or $+2$.}
        \item If $a, b \in A$, then $a * b \in A$. \footnote{$A$ is closed under the operation $*$}
    \end{itemize}
\end{defns}

    
\begin{defns}[Commutativity]
    An operation $*$ is said to be \emph{commutative} if it satisfies
    \begin{align}
        a * b  = b * a
    \end{align}
    for any two elements $a$ and $b$ in $A$.
\end{defns}

\begin{defns}[Associativity]
    An operation $*$ is said to be \emph{associative} if it satisfies 
        \begin{align}
            (a * b) * c = a * (b * c)
        \end{align}
        for any three elements $a$, $b$ and $c$ in $A$.
\end{defns}

\begin{defns}[Identity element]
    The \emph{identity} element $e$ with respect to the operation $*$ has the property that:
        \begin{align}
            e * a  = a \quad\text{ and }\quad a * e = a
        \end{align}
        is true for every element $a$ in $A$.
\end{defns}

\begin{defns}[Inverses]
    The inverse of any element $a$, item denoted by $a^{-1}$ has the property that:
        \begin{align}
            a * a^{-1} = e  \quad\text{ and }\quad a^{-1} * a = e
        \end{align}
\end{defns}

\section{The Definitions of Groups}

\begin{quest}
    What is a \textbf{group}?
\end{quest}

\begin{defns}[Informal definition]
    A group is defined to be a set with an operation ($*$) which is associative, has an identity element, and each element in the set has an inverse.
\end{defns}

\begin{defns}[Formal definition]
    A group is a set $G$, together with an operation $*$ which satisfies:

    \begin{itemize}
        \item $*$ is associative.
        \item There exists an element $e$ in $G$ such that $a*e = a$ and $e*a=a$ for every element $a$ in $G$.
        \item For every element in $a$ in $G$, there is an element $a^{-1}$ in $G$ such that $a * a^{-1} = e$ and $a^{-1} * a = e$.
    \end{itemize}
\end{defns}

\noindent
A group as defined above is usually denoted by the pair symbol $(G,*)$, which denotes that a group is a set $G$ together with the operation $*$. \footnote{If there is no chance of ambiguity, the group is usually denoted with just the letter $G$.}

\marginnote{
    \begin{rem}
        The set of integers $\mathbb{Z} = \{\ldots,-2, -1, 0 , 1, 2, \ldots\}$ is a group with the operation of addition, denoted by $(\mathbb{Z}, +)$. Similarly,the set of rationals numbers and addition $(\mathbb{Q}, +)$, and the set of real numbers $(\mathbb{R}, +)$.
    \end{rem}
}

\marginnote{
    \begin{rem}
        Many a times, algebraic structures apparent in the study of natural phenomena (that is to say in physics) are groups,~\emph{i.e.} quantum spin, angular momentum
    \end{rem}
}

\begin{exmp}[Finite groups: Groups of integers modulo $n$]
    The group of integers modulo $n>1$ consists of the set
    \begin{align}
        \{0, 1, 2, \ldots, n - 1\}
    \end{align}

    \noindent
together with the operation of addition modulo $n$; The addition of two numbers $a$ and $b$ modulo $n$, can be described by imaging a set of equidistant points on an arc of a unit circle. To add $a$ and $b$, we start at $a$ and hop $b$ points on the arc each at an angle of $2\pi/n$ from the next, where we end up will be the sum $a + b$, see \refFigureOnly{modulo_n_circle}. This operation is associative (instead of starting at $a$, we can start at $b$ and hop $a$ times, we'll end up at $a + b$ again). The identity element for this group is $0$, and the $n - a$ is the inverse of $a$ ($a + n - a = n = 0$). \footnote{The element $-a$  would seem to quality as inverse of $a$, $a + (-a) = 0$. However $-a$ does not belong to the group.} Such a group is denoted by the symbol $\mathbb{Z}_{n}$.

\emph{Cayley} table shows the operation of a finite group, by arranging all possible group operations of all the elements in the group in a square table, and from the Cayley table, many properties of the group can be easily discerned. Consider the Cayley table for the group $\mathbb{Z}_3$ below:

\[
    \begin{array}{l|*{3}{l}}
       +   & 0   & 1   & 2 \\
    \hline
       0   &  0  &  1  & 2\\
       1   &  1  &  2  & 0 \\
       2   &  2  &  0  & 1 \\
    \end{array} 
\]

\noindent
a quick glance at the table, we can see that $\mathbb{Z}_3$ is a commutative or Abelian group and $1$ and $2$ are inverses of one another. Any finite group $(G, *)$ has a Cayley table of the form

\[
    \begin{array}{l|*{4}{c}}
        *   & \cdots   &  y  & \cdots \\
    \hline
       \vdots   &     &   &     \\
       x        &     & x*y   &     \\
       \vdots   &     &   & \\
    \end{array} 
\]

\noindent
each element in $G$ has one designated row and similarly a column, then the entry in the row of $x$ and the column of $y$ is $x * y$.


\begin{marginfigure}
    \centering
    \def\svgwidth{\linewidth}
    \resizebox{0.6\textwidth}{!}{\input{graphics/modulo_n_circle.pdf_tex}}
    \caption{Addition modulo $n$ can be visualized by hoping around equidistant points on an arc of a unit circle.}
    \labelFigure{modulo_n_circle}
\end{marginfigure}

\end{exmp}

\end{document}
