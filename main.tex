\documentclass[nohyper,nobib]{tufte-handout}

\usepackage{graphicx} % allow embedded images
  \setkeys{Gin}{width=\linewidth,totalheight=\textheight,keepaspectratio}
  \graphicspath{{graphics/}} % set of paths to search for images
\usepackage{amsmath,amsfonts,amsthm, amssymb, physics} % Math packages
\usepackage[english]{babel} % English language/hyphenation
\usepackage{booktabs} % book-quality tables
\usepackage{units}    % non-stacked fractions and better unit spacing
\usepackage{multicol} % multiple column layout facilities
\usepackage{lipsum}   % filler text
\usepackage{fancyvrb} % extended verbatim environments
\fvset{fontsize=\normalsize}% default font size for fancy-verbatim environments
\usepackage{fancyhdr} % Custom headers and footers
\pagestyle{fancyplain} % Makes all pages in the document conform to the custom headers and footers
\fancyhead{} % No page header - if you want one, create it in the same way as the footers below
\fancyfoot[L]{} % Empty left footer
\fancyfoot[C]{} % Empty center footer
\fancyfoot[R]{\thepage} % Page numbering for right footer
\renewcommand{\headrulewidth}{0pt} % Remove header underlines
\renewcommand{\footrulewidth}{0pt} % Remove footer underlines
\setlength{\headheight}{13.6pt} % Customize the height of the header
\allowdisplaybreaks
\usepackage{mathpazo}
\usepackage[activate={true,nocompatibility},final,tracking=true,kerning=true,spacing=true,factor=1100,stretch=10,shrink=10]{microtype}
\usepackage{float}
\usepackage{caption}
\usepackage{verbatim}
\usepackage{kantlipsum} % Used for inserting dummy 'Lorem ipsum' text into the template
\usepackage{sectsty} % Allows customizing section commands
\allsectionsfont{\normalfont \bfseries} % Make all sections centered, the default font and small caps
\usepackage{enumerate}
\usepackage{color}
\usepackage[dvipsnames]{xcolor}
\usepackage{xspace}
\usepackage{float}
\usepackage[colorlinks=true, linkcolor=WildStrawberry, urlcolor=WildStrawberry, citecolor=Emerald, linktocpage=true, breaklinks=true]{hyperref}
%--------Theorem Environments--------
%theoremstyle{plain} --- default
\newtheorem{thm}{Theorem}
\newtheorem{cor}[thm]{Corollary}
\newtheorem{prop}[thm]{Proposition}
\newtheorem{facts}[thm]{Facts}
\newtheorem{fact}[thm]{Fact}
\newtheorem{clm}[thm]{Claim}
\newtheorem{lem}[thm]{Lemma}
\newtheorem{conj}[thm]{Conjecture}
\newtheorem{quest}[thm]{Question}

\theoremstyle{definition}
\newtheorem{defn}[thm]{Definition}
\newtheorem{defns}[thm]{Definitions}
\newtheorem{con}[thm]{Construction}
\newtheorem{exmp}[thm]{Example}
\newtheorem{exmps}[thm]{Examples}
\newtheorem{notn}[thm]{Notation}
\newtheorem{notns}[thm]{Notations}
\newtheorem{addm}[thm]{Addendum}
\newtheorem{exer}[thm]{Exercise}

\theoremstyle{remark}
\newtheorem{rem}[thm]{Remark}
\newtheorem{rems}[thm]{Remarks}
\newtheorem{warn}[thm]{Warning}
\newtheorem{sch}[thm]{Scholium}


\geometry{
	left=13mm, % left margin
	textwidth=130mm, % main text block
	marginparsep=8mm, % gutter between main text block and margin notes
	marginparwidth=55mm % width of margin notes
}

\fontsize{10}{20}\selectfont

%----------------------------------------------------------------------------------------
%	TITLE SECTION
%----------------------------------------------------------------------------------------

\title{Notes on Charles Pinter's book of abstract algebra} % The assignment title
\author{Unathi Skosana}
\date{\vspace{-5pt}\normalsize \today} % Today's date or a custom date

\begin{document}
\justifying 
\maketitle

\begin{abstract}
\noindent
These are notes taken while reading Charles Pinter's 'A Book Of Abstract Algebra`
\end{abstract}

\tableofcontents

\section{Operations}

\begin{quest}
	What is an \textbf{operation} on a set A?
\end{quest}

\begin{defns}[Informal definition]
    An operation is any rule which assigns to each ordered pair of elements of $A$ a unique element in $A$.
\end{defns}

\begin{defns}[Formal definition]
	Let $A$ be any set:\newline An operation $*$ on A is a rule which assigns to each ordered pairs $(a,b)$ of elements of A exactly one $a * b$ in A
    \begin{itemize}
        \item $a * b$ is defined for \emph{every} ordered pair $(a,b)$ of elements of $A$. \footnote{In $\mathbb{R}$, division does not qualify as operation since it does not satisfy this condition. i.e. the ordered pair $(a, 0)$ has undefined quotient $a / 0$.}
        \item $a * b$ must be \emph{uniquely} defined. \footnote{If $\diamond$ is defined on $(a, b)$ to be the number whose square is $ab$. In $\mathbb{R}$, $\diamond$ does not qualify as an operation since $2 \diamond 2$  could be either $2$, or $+2$}
        \item If $a, b \in A$, then $a * b \in A$. \footnote{$A$ is closed under the operation $*$}
    \end{itemize}
\end{defns}

\begin{defns}[Superfluous properties]
    \begin{itemize}
        \item An operation $*$ is said to be \emph{commutative} if it satisfies
            \begin{align}
                a * b  = b * a
            \end{align}
            for any two elements $a$ and $b$ in $A$.
        \item An operation $*$ is said to be \emph{associative} if it satisfies 
            \begin{align}
                (a * b) * c = a * (b * c)
            \end{align}
            for any three elements $a$, $b$ and $c$ in $A$.
        \item The \emph{identity} element $e$ with respect to the operation $*$ has the property that:
            \begin{align}
                e * a  = a \quad\text{ and }\quad a * e = a
            \end{align}
            for every element in $a$ in $A$.
        \item The inverse of any element $a$, denoted by $a^{-1}$ has the property that;
            \begin{align}
                a * a^{-1} = e  \quad\text{ and }\quad a^{-1} * a = e
            \end{align}
    \end{itemize}
\end{defns}

\end{document}
